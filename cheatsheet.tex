\documentclass[10pt]{extarticle}
\usepackage{extsizes}

\usepackage[letterpaper, margin=0.5in]{geometry}
\usepackage[table]{xcolor}
% Get rid of paragraph indentation
\usepackage{parskip}
\usepackage{paracol}
\usepackage{enumerate}
\usepackage{tabularx}
\usepackage{longtable}
%nicer fonts
\usepackage[T1]{fontenc}
\usepackage{inconsolata}

% Setup colors
%\definecolor{DarkBackground}{HTML}{0F420F}
%\definecolor{DarkBackgroundText}{HTML}{FFFFFF}
\definecolor{DarkBackground}{HTML}{DDDDDD}
\definecolor{LightBackground}{HTML}{D6DED6}

% Increase space between table rows
\renewcommand{\arraystretch}{1.5}

% Setup page headers and footers
\usepackage{fancyhdr}
\pagestyle{fancy}
\setlength{\headheight}{14.0pt}
\lhead{\large Fish Shell Cheat Sheet}
\chead{}
\rhead{For Fish 2.1.1; updated \today}
\lfoot{}
\cfoot{\textcopyright Copyright 2015 Michael Younkin}
\rfoot{\thepage}

\begin{document}
\begin{paracol}{2}

\section*{Syntax}

\subsection*{Variables}

Three kinds: universal, global, and local. Universal variables are shared btw.
all sessions on the computer per user. Global variables are specific to the
current fish session, but they are outside of any block scope. Local variables
are specific to a particular block scope and are automatically erased.

Set a variable as universal with \texttt{-U}, as global with \texttt{-g}, or
local with \texttt{-l}.

Scoping rules are as follows:

\begin{enumerate}

\item If a variable is explicitly set to either universal, global or local, that
setting will be honored. If a variable of the same name exists in a different
scope, that variable will not be changed.

\item If a variable is not explicitly set to be either universal, global or
local, but has been previously defined, the variable scope is not changed.

\item If a variable is not explicitly set to be either universal, global or
local and has never before been defined, the variable will be local to the
currently executing function. Note that this is different from using the -l or
–local flag. If one of those flags is used, the variable will be local to the
most inner currently executing block, while without these the variable will be
local to the function. If no function is executing, the variable will be global.

\end{enumerate}

\subsection*{Exporting Variables}
Export a variable with \texttt{set -x}.

\subsection*{Arrays}
Store multiple strings in one variable with an array.
\subsubsection*{Access an index}
\texttt{echo \$PATH[3]}
\subsubsection*{Iterate}
\begin{verbatim}
for i in $PATH
    echo $i in the path
end
\end{verbatim}
\subsubsection*{Definition}
Make an array called smurf containing "blue" and "small":\\
\texttt{set smurf blue small}
\subsubsection*{Delete an element}
\texttt{set -e smurf[1]}

\subsection*{Functions}
Define a function like so:\\
\begin{verbatim}
function ll
    ls -l \$argv
end
\end{verbatim}
Access arguments using \texttt{\$argv}, call the function using \texttt{ll}.

\subsection*{Jobs}
When you execute a command, it starts a job. You can put a job in the background
by adding the \texttt{\&} suffix. You can suspend a currently running job using
\texttt{Ctrl-Z}. You can put the suspended job in the background with
\texttt{bg}. Finally, you can list all running jobs with \texttt{jobs}.

\subsection*{Chaining Commands}
Each command ends in either a newline or a semicolon. Chain commands using
\texttt{command1; and command2} or \texttt{command1; or command2}. \texttt{and}
and \texttt{or} check the previous command's exit status and act accordingly.

\subsection*{Aliases}
To define an alias, either make a function or use \texttt{alias NAME
DEFINITION}, which actually just defines a function for you.

\switchcolumn

\section*{Built-in Variables}

\rowcolors{1}{LightBackground}{white}
\begin{tabularx}{\columnwidth}{X X}
    \texttt{argv} & array of arguments to a shell function\\
    \texttt{history} & array containing the command history\\
    \texttt{HOME} & the user's home directory\\
    \texttt{PWD} & the current working directory\\
    \texttt{status} & the exit status of the last foreground job to exit\\
    \texttt{USER} & the current username \\
    \texttt{PATH} & a global variable automatically reset in each new fish session
\end{tabularx}


\section*{IO Redirection and Piping}

\rowcolors{1}{LightBackground}{white}
\begin{tabularx}{\columnwidth}{X X}

    Redirect stdin & \texttt{N<SOURCE\_FILE} (N is optional, default is 0) \\
    Redirect stdout & \texttt{N>DESTINATION} (N is optional; default is 1) \\
    Redirect stderr & \texttt{N\^{}DESTINATION} (N is optional; default is 2) \\
    Redirect with appending & \texttt{>>} or \texttt{\^{}\^{} + DESTINATION\_FILE} \\
    Close FD & use \texttt{-} as \texttt{SOURCE\_FILE} or \texttt{DESTINATION} \\
    Pipe stdout & \texttt{command1 | command2} \\
    Pipe a different FD & \texttt{command1 N>| command2} \\

\end{tabularx}

\section*{Expansion}

\subsection*{Support for Expansion in Quotes}

\rowcolors{1}{LightBackground}{white}
\begin{tabularx}{\columnwidth}{X X X}
    \rowcolor{DarkBackground}
    \textbf{Type} & \textbf{Var Exp?} & \textbf{Esc. Char} \\
    none & Yes & All \\
    " " & Yes & \textbackslash", \textbackslash\$, and \textbackslash\textbackslash \\
    ' ' & No & \textbackslash', \textbackslash\textbackslash
\end{tabularx}

\subsection*{Command Expansion}
Surround command in parentheses. If it returns multiple lines, they'll be joined
with spaces.

\subsection*{Parameter Expansion}
Fish supports more limited globbing than other shells; use \texttt{find} with
command expansion for more complicated globs. Files beginning with . are ignored
unless . is the first character in the glob.

\rowcolors{1}{LightBackground}{white}
\begin{tabularx}{\columnwidth}{X X X}
    \rowcolor{DarkBackground}
    \textbf{Char} & \textbf{Behavior} & \textbf{Exception} \\
    ? & any single character & / \\
    * & any string of characters & / \\
    ** & any string of characters & none
\end{tabularx}

\subsection*{Brace Expansion}
Same as in bash.\\
\texttt{echo input.{c,h,txt}}\\
\texttt{>> input.c input.h input.txt}

\subsection*{Variable Expansion}
A \$ followed by a string of characters is expanded to the value of the
environmental variable with that name. Surround with braces to separate from
text.

\subsection*{Process Expansion}
\% followed by a string is expanded into a PID according these rules:

\begin{enumerate}
    \item If the string is \texttt{self}, insert the shell PID
    \item If the string is the ID of a job, insert the process group ID for
        the job
    \item If any child processes match the string, insert their PIDs
    \item If any processes owned by the user match the string, insert their PIDs
    \item else produce an error
\end{enumerate}

\subsection*{Index Range Expansion}
Select a range of values from an array using ..:\\
\texttt{echo (seq 10)[2..5 1..3]}\\
\texttt{>> 2 3 4 5 1 2 3}

\switchcolumn

\section*{Editor Shortcuts}

\begin{tabularx}{\columnwidth}{X >{\ttfamily}X}
    Complete current token & Tab \\
    Accept autosuggestion & at EOL: End/Ctrl-E/Right/Ctrl-F \\
    Move to BOL & Home/Ctrl-A \\
    Move to EOL & End/Ctrl-E \\
    Move characterwise & Left/Ctrl-B or Right/Ctrl-F \\
    Move wordwise & Alt-Left or Alt-Right \\
    Move through directory listing & on empty CMD line: Alt-Left or Alt-Right \\
    Search history for prefix in CMD line & Up or Down \\
    Search history for token containing token under cursor & Alt-Up or Alt-Down \\
    Delete characterwise & Delete/Ctrl-D (forwards) or Backspace (backwards) \\
    Delete entire line & Ctrl-C \\
    Move contents from cursor to EOL to killring & Ctrl-K \\
    Move contents from BOL to cursor to killring & Ctrl-U \\
    Repaint Screen & Ctrl-L \\
    Move previous word to killring & Ctrl-W \\
    Move next work to killring & Alt-D \\
    Print description of CMD under cursor & Alt-W \\
    List contents of current directory or directory under cursor & Alt-L \\
    Add '|less;' to end of job under cursor & Alt-P \\
    Capitalize current word & Alt-C \\
    Make current word uppercase & Alt-U
\end{tabularx}

\end{paracol}
\end{document}
